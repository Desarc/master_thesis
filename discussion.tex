\chapter{Discussion and Conclusion}
\label{ch:discussion}
The work of this thesis has been focused around creating a better environment for learning about software modeling languages, more specifically focusing on \gls{uml} activities within the context of the \emph{Reactive Blocks} modeling tool. Inspired primarily by methods used to teach players how to play video games, two different learning environments were explored.

\section{The Tutorial}
\label{sec:discussion_tutorial}
The first teaching method explored was the \emph{tutorial}, a concept widely used to teach various topics by offering interactive introductions, often with exercises. This approach to teaching is also used in many video games, where players are exposed to a tutorial that teaches them the basics of the game before they are allowed to explore the game on their own. The design, implementation, testing, and evaluation of the tutorial is covered in Ch.~\ref{ch:reactive_blocks_tutorial}.

\noindent
When creating the tutorial, various learning practices were taken into consideration. These were a mixture of good practices for video game tutorials and other types of tutorials, from both formal and informal sources. A summary of these is provided in Sect.~\ref{sec:good_practices_tutorials}.

\noindent
A few iterations of going through \gls{uml} activity concepts and figuring out their interdependencies, finding the least amount of information required to describe them accurately, and designing suitable exercises for understanding their semantics, resulted in a seven-step tutorial. Each step gave a short introduction of one or more concepts, and challenged learners with an exercise to solve using these concepts. The tutorial and its seven steps is described in more detail in Sect.~\ref{sec:tutorial_design_implementation}.

\noindent
Unfortunately, not all of these practices were feasible to implement within the given frame, particularly because of the constrains provided by the proprietary Reactive Blocks modeling environment. This meant some of the goals that had initially been set for the tutorial had to be forgone, and some adjustments had to be made to the tutorial design. The most prominent change was that all concept and task information had to be provided outside the modeling environment, in a supplementary document. Sections~\ref{sec:tutorial_goals_fulfilled} and~\ref{sec:tutorial_goals_not_fulfilled} provide a more detailed overview of the extent to which each goal was fulfilled by the tutorial design and implementation.

\noindent
In order to uncover any major flaws or issues with the tutorial, and additionally get an impression of its teaching potential, a user test with a small number of test subjects was conducted. The testing session revealed a number of issues that required some additional consideration, and the main bulk of these were related, directly or indirectly, to the use of Reactive Blocks as the modeling environment. The session also provided some indications that the tutorial could indeed be a very good way of learning about \gls{uml} activities, particularly with respect to providing motivation for learners. The tutorial may not be an adequate tool for learning to use Reactive Blocks however, as follow-up observations of the test subjects revealed that many of them struggled when working with Reactive Blocks, on account of the concepts that were not covered in the tutorial.

\noindent
Although I was unable to implement and test several of the tutorial design principles, the tutorial served as a good starting point for establishing a teaching order for concepts in \gls{uml} activities, and figuring out how much information users need in order to solve certain types of exercises related to each concept. These results laid the foundation for the \emph{Reactive Blocks game}, which was the second teaching method explored.

\section{The Game}
\label{sec:discussion_game}
The second teaching method explored in this thesis was to teach \gls{uml} activity concepts inside a game. Learning games are becoming very popular in many levels of education because they offer a more immersive and interactive learning experience, among other things. Teaching \gls{uml} activities with a game gave more potential for utilizing game strategies for introducing and teaching new concepts, such as giving learners a sense of \emph{empowerment} and \emph{identity}. The design, implementation, testing, and evaluation of the learning game is covered in Ch.~\ref{ch:tutorial_game}.

\noindent
Like with the tutorial, the learning game is based on a set of principles for designing this type of game, and these are summarized in Sect.~\ref{sec:good_practices_games}. Additionally, the game would have a lot in common with a tutorial, so consideration was made with respect to the principles for good tutorials, and the lessons learned from the design and testing of the tutorial in Ch.~\ref{ch:reactive_blocks_tutorial}.

\noindent
The design and implementation process resulted in a first prototype version of a game with 5 levels, covering only part of the seven steps from the tutorial. The game was meant to have more levels, but we wanted to conduct a usability study in order to discover any serious issues as early as possible. The concept of the game is a two-dimensional world, where the player controls a character, performing various tasks. The character is controlled entirely through \gls{uml} activity/Reactive Blocks models, which the player has to set up prior to launching the game level. Each level also has an introduction part, similar to the introductions given with the tutorial, but more extensive and with additional resources. The complete design and implementation of the game is covered in Sections~\ref{sec:game_design} and~\ref{sec:game_implementation}.

\noindent
While the game format allowed consideration and implementation of some additional ``good'' practices, there were still issues that were difficult to deal with because of the Reactive Blocks modeling environment. A usability test with three test subjects revealed that all three, who were novice users, had difficulties with correctly understanding and using a number of \gls{ui} processes and elements in Reactive Blocks. This was despite the fact that some additional \gls{ui} help had been added to the introduction part of the game, as a result of some of the same issues appearing in the tutorial test. Apart from those related to Reactive Blocks, the usability test revealed only minor \gls{ui} issues that could be taken into consideration for the next version of the prototype.

\noindent
In addition to usability-related results, the testing sessions also revealed that the test subjects gained a lower understanding of the concepts presented than desired. Some hints of potential misconceptions were also observed. This could mean that 5 levels is not sufficient to really learn these concepts, and that more practice is required. If this is the case, it can likely be solved by adding additional levels and challenges, for the players to practice their skills and knowledge more, and with different perspectives. In any case, more testing is needed to properly verify and understand the issue, preferably on a more complete version of the game.

\noindent
On a brighter note, all three subjects stated that they enjoyed the game as a tool for learning about \gls{uml} activities and Reactive Blocks, and would likely prefer it over other ways of learning. However, more than three opinions are likely needed in order to verify a real interest in this type of learning resource. It also remains to be seen whether a game like this will instill deeper learning in players, but research supports that this is the case for well-designed educational games (see Sect.~\ref{sec:game_based_learning}).

\noindent
The game is still in a relatively early stage, but the results so far are encouraging. Further development of the game will however not continue within the scope of this thesis, because of time constraints.


\section{Concluding Remarks}
\label{sec:concluding_remarks}
The work of this thesis has been about exploring the use of game-related learning principles and strategies for teaching software modeling with \gls{uml} activities. Two different teaching approaches were explored:  the first approach simply incorporated many of the learning principles in a stepwise modeling tutorial, and the second approach involved designing a game with levels teaching the same steps, but in a more immersive environment.

\noindent
Some minor tests of the two approaches were conducted, yielding mostly indicative but interesting results. Both approaches were found to be flawed in some ways, but they also showed great potential for teaching the concepts in question to the target audience, in an interesting and motivating way. It is however difficult to draw any conclusions about how the two approaches would measure up against other learning resources for the same concepts, without data from comparative studies.

\noindent
It is also worth noting that with regard to established theoretical usability and learning principles, both approaches measure up quite well by \emph{design}, despite the \emph{implementations} lacking some important features. Whether implementing all of these features will actually be beneficial to the learning experience or not is however a little uncertain, as users have been known to behave unpredictably~\cite{andersen:tutorials_impact}.

\noindent
In the end, there is no reason to believe that learning strategies from games will \emph{not} be appropriate also for learning modeling languages, as there have been no results indicating this. At the same time, there is not enough ground in this thesis alone to conclude that they \emph{definitely will be} appropriate either. There are however several other research efforts endorsing this approach for teaching programming, which is a very closely related topic (see Sect.~\ref{sec:learning_with_games}).

\noindent
The conclusion to be drawn from this thesis is the following: using learning strategies from games to teach modeling languages is definitely worth exploring, but designing and implementing a good learning environment is not trivial, and requires a considerable amount of work and effort. Hopefully, the work and examples provided here can serve as a starting point for others.

\clearpage

\section{Future Work}
\label{sec:future_work}
As the final part of this thesis, some suggestions for future work are included. There are several possible paths to take, and some are described in the following sections.

\subsection{Complete Game}
The most prominent path to take would be to continue development of the Reactive Blocks game, improving on the prototype and moving closer to a ``finished'' game. The current prototype is a little lacking in content, in addition to needing some minor fixing. There is also great potential for improving the modeling environment, particularly in order to improve usability. Section.~\ref{sec:game_improvement} contains some more concrete suggestions on how to continue the development of the game. 

\subsection{Comparative Studies}
The weakest part of this thesis is likely that I was unable to perform proper comparative studies, pitting the \emph{Reactive Blocks game} against other similar learning resources in order to measure factors like player engagement and levels of understanding. This was not feasible because of constraints on time and resources. Comparative studies would however have been prudent in this context, and should thus be the second priority for any future work (after fixing the simpler issues and adding more content to the game).

\noindent
More specifically, the game can be studied in comparison to the tutorial in Ch.~\ref{ch:reactive_blocks_tutorial}, any of the existing tutorials mentioned in Sect.~\ref{sec:existing_tutorials}, or the lazier approach of just letting learners dig through documentation.

\subsection{Other Modeling Languages}
This thesis has been focused around teaching \gls{uml} activities, more specifically in the context of Reactive Blocks, using teaching principles and strategies from games. It could also be interesting to use the same approach with other modeling languages, such as \gls{uml} state machines, sequence diagrams, flowcharts, or even class diagrams. These will likely provide different challenges with respect to game and exercise design, but allow use of the same teaching principles.