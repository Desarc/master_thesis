\chapter*{Abstract (Norsk)}
Å lære et språk for software-modellering fra dens dokumentasjon kan være både vanskelig og forvirrende, og mange eksisterende \emph{tutorials} for slike språk er bare marginalt bedre. Samtidig bruker mennesker som spiller videospill timesvis på å lære spill som krever mestring av komplekse strategier og konsepter, uten å miste hverken interessen eller motivasjonen. Disse spillenes evne til å gi motivasjon for læringsinnsats støtter en voksende trend for læringsspill, en type spill ment for å lære bort ulike emner til mennesker i alle aldre. Denne avhandlingen utforsker prinsippene og strategiene som gjør videospill til gode læringsplattformer, og gjør et forsøk på å bruke disse prinsippene til å lære bort software-modellering på en engasjerende måte.

\noindent
Avhandlingen fokuserer på modellering med \emph{\gls{uml} activities} i kontekst av Reactive Blocks, og presenterer to forskjellige tilnærminger for å lære bort de ulike konseptene det innebærer. Den første tilnærmingen er en forbedret tutorial, som bruker prinsipper om interaktivitet og å gi informasjon og instruksjon i riktig kontekst for å engasjere studenter. Den andre tilnærmingen er et læringsspill, som i tillegg lar studentene visualisere konseptene og fordype seg i læringsomgivelsene.

\noindent
Avhandlingen beskriver i detalj design og implementasjon av prototyper for både den interaktive tutorialen og læringsspillet, og prinsippene de er basert på. Begge proto\-typene evalueres også med hensyn til disse prinsippene, med fokus på brukervennlighet og potensiale som læringsverktøy, og med støtte i data fra brukertester.