\chapter{A Tutorial Game}
\label{ch:tutorial_game}
As the second part of my own work, I designed and implemented a prototype for an educational game for \gls{uml} Activities in the context of Reactive Blocks, nicknamed simply \emph{The Reactive Blocks Game}. Based on the tutorial described in Ch.~\ref{ch:reactive_blocks_tutorial}, this game teaches the same concepts in similar ways, but in a different environment. This chapter covers the motivation for creating the game, as well as a description of the design and implementation of the prototype. Additionally, a user test was conducted in an effort to uncover usability issues with the game prototype.

\section{Motivation}
\label{sec:game_motivation}
The target audience for the game is the same as for the tutorial, and described in Sect.~\ref{sec:tutorial_target_audience}.
TODO


\subsection{Goals}
\label{sec:game_goals}
Like with the tutorial in Ch.~\ref{ch:reactive_blocks_tutorial}, I initially set some goals for the design and implementation of the game. These goals are primarily based on experiences from the tutorial implementation and test, as well as the good practices from Sect.~\ref{sec:good_practices_games}.

\paragraph{Immersion} The primary goal for the game is to provide an environment where players can immerse themselves in the learning experience. This involves giving the players a sense of identity within the game, and providing them with challenges that are easily visualized and comprehended (but not necessarily solved) within the given environment.

\paragraph{Exploration and Guidance} \gls{uml} Activities can be used to model very complex systems, and learning how to do this is not a trivial process. The game is thus likely to benefit from providing some guidance to players, as opposed to complete freedom to explore~\cite{andersen:tutorials_impact}. Some freedom is still necessary, to facilitate creativity and deeper learning~\cite{bonawitz:double_edged_pedagogy}. The goal is then that the game should be primarily focused around a tutorial-like guided path for learning about concepts within \gls{uml} Activities, but with the possibility for players to find their own solutions to problems. Players should also be encouraged to explore other perspectives, and maybe find different or even \emph{better} solutions to the same problems.

\paragraph{Level of Difficulty} The target audience for the game is people who want or need to learn about software modeling with \gls{uml}, which is likely to be a fairly small group of people (in the big picture). However, these people often come with different backgrounds: some may be experienced programmers looking for a way to visualize their programs, while others are people without any previous experience dealing with software structures. The game should be easy enough for the less knowledgeable users to get started without frustration, but eventually provide challenges that feel relevant also to the more experienced. Players should also have the option to skip challenges they feel are less relevant or that involve concepts they have already mastered, in order to avoid the game experience becoming tedious.

\paragraph{Lessons from the Tutorial} The design and implementation of the tutorial in Ch.~\ref{ch:reactive_blocks_tutorial} fulfilled some, but not all of the goals that were set. The approach seemed to provide a decent learning experience, but with some potential for improvement. A goal for the game is to build on the parts of the tutorial that were successful, such as the teaching order, and additionally include some of the parts that were not implemented, such as having everything in one place. Since the game still will be based on Reactive Blocks, many of the same challenges will be present, but with a game, there should be more options for providing resources within the same platform, such as help-on-demand.

\section{Concept Development}
\label{sec:game_concept}
When developing the concept of the game, a few points were considered. First of all, it should involve some sort of main character, giving the player a sense of identity. Secondly, the game concept must accommodate a range of different types of challenge that are suited for learning about concepts like concurrency, modularization and reuse. Finally, it must be relatively easy to prototype, as the available time to implement the game was fairly limited.

\subsection{The Final Concept}
\label{sec:game_final_concept}
With the above design points in mind, I settled on a two-dimensional maze-navigating game, where the player controls a character or set of characters through operations and logic in Reactive Blocks. The main character, Malcolm, must perform one or more goals to complete the level, such as gathering stars. In order to get to these stars however, Malcolm must perform some other tasks, such as gathering keys, or invoke the assistance of other characters in the game.

\noindent
This concept was chosen for the following reasons:
\begin{itemize}
	\item Two-dimensional games are quick and easy to implement, both graphically and programming-wise, given an appropriate framework.
	\item Since the purpose of the game is to teach \gls{uml} Activities, it makes most sense to provide a game environment where the player must program their characters' path in advance, and then see if they modeled the desired behavior correctly. A two-dimensional maze-navigating game makes it easy for players to get a clear overview of what they are supposed to do.
	\item Despite being relatively simple to implement, a two-dimensional maze-game offers a lot of possibilities in the form of obstacles and challenges the player must overcome in order to complete each level.
\end{itemize}

\noindent
Even with such a flexible concept in place, it was no trivial task to create levels and challenges that let the user learn about \gls{uml} Activity concepts in an intuitive and reasonable way. In addition, I had to consider how each concept should be introduced within the level, so that the user would know what to expect.

\subsection{The Tutorial Part}
\label{sec:game_tutorial}
One of the most important goals for the game was to provide an environment where the player could find all the information needed to complete a level, instead of having to switch between contexts or search through external sources. Players would be creating the logic for each level in Reactive Blocks, generate code from these models, and run it to see if it was correct. Thus, additional information, such as concept introductions, level maps, and help-on-demand, had to be presented inside Reactive Blocks. Unfortunately, because I was unable to alter the Reactive Blocks environment, I could still not provide a tutorial mode, just as with the tutorial in Ch.~\ref{ch:reactive_blocks_tutorial}.

\noindent
Instead of a tutorial mode in Reactive Blocks, a different solution presented itself. I could create a separate program with a window that would contain the most basic information needed for each level, with internal links to additional resources. This would of course not be completely within the context of Reactive Blocks, but hopefully a decent substitute for the lack of a real tutorial mode.

\noindent
Designing the tutorial window presented some challenges of its own. The window needed to be small enough for players to be able to keep it on their screen together with the Eclipse/Reactive Blocks window, preferably also for smaller screens like on laptops. At the same time, I needed to include all the info a player would realistically need to complete each level. Obviously I would not be able to fit all the information in one small window at the same time, so I had to decide which elements to display by default, and what should only be displayed upon the player's request.

\noindent
After some consideration, I settled on a 600x800 pixels window, which should fit pretty good with the Reactive Blocks modeling canvas on screen resolutions down to 1280x800. When the resolution becomes smaller than this, there is little room for anything but the Eclipse window anyway. Given more time for development, it would likely be better to have a variable size window depending on the available screen resolution, but for now, this size is fixed.

\noindent
Figure~\ref{fig:level1_intro_mockup} shows a mock-up of the introduction window design. The layout is the same for all levels, but the content varies. At the top is simply the title of the level, with a ``tagline'' giving a short description of what the current level is about. Below the title is a slide show, which the player may go through to get a quick introduction of the concepts taught by, and the goals of, the current level. These introductions are generally kept short, except for in level 1, where the whole game must be introduced in additional to all the most basic concepts. The player controls the slide show with the arrow and play/pause buttons.

\begin{figure}[htp]
	\centering
	\includegraphics[scale=0.50]{level1_intro_mockup}
	\caption[Introduction window mock-up]{The introduction window design, with the introduction slide show at the top, followed by the goals for this level, a button for tips, the level map, and some buttons for retrieving additional information about the concepts presented.}
	\label{fig:level1_intro_mockup}
\end{figure}

\noindent
Below the slide show, we see a list of the goals for the current level. The goals, together with the level map at the bottom, is the most important information of the level. The level map and the goals are always visible, so that the user has easy access to these when working on the solution for the current level. The level map varies in size, but is always scaled to fit the small frame in the bottom left corner. However, since the player sometimes has to check details of the map, it is possible to make it bigger by clicking on it.

\noindent
Finally, the buttons on the right side of the window represent the \emph{help-on-demand} information that is not visible by default. The \emph{Tips} button provides some tips that are good to know either for solving the current level, or for learning about modeling in general. The hope is that if players get stuck or are unsure about how to proceed, they will see and click this button in an attempt to find help. The \emph{Tell me more about...} buttons provide more detailed information about each concept presented in the current level, available for players who need more information in order to understand the concepts sufficiently, or simply want to learn more.

\subsection{Game Levels}
\label{sec:game_levels}
For the first version of the prototype, only 5 levels were designed and implemented. This was because I wanted to test the prototype with a few users as soon as possible, in order to uncover any serious flaws with the \gls{ui} or the way the introductions were presented as early as possible. 

\noindent
The prototype levels roughly follow the teaching order described in Sect.~\ref{sec:teaching_order}. The first version of the prototype covers steps 1 through 3, with level 5 introducing part of step 4. The game levels are briefly described below.

\subsubsection{Level 1}
Figure~\ref{fig:level1map} displays the map for level 1. The purpose of this level is to introduce the game concept, the main character Malcolm, and the most basic modeling concepts in Reactive Blocks. The goal is pretty much identical to exercise 1 of the tutorial (create a \emph{Hello World!} program), but the game additionally visualizes this task through the game character. The player is provided with the \emph{sayHello} operation, which makes a speech bubble appear next to the character on the level map.

\begin{figure}[htp]
	\centering
	\includegraphics[scale=0.40]{level1map}
	\caption[Level 1 of the Reactive Blocks game]{Level 1 of the Reactive Blocks game. The player must make the character speak, which makes a speech bubble appear beside him.}
	\label{fig:level1map}
\end{figure}

\noindent
\textbf{Goal:} Make the game character speak the famous words ``Hello world!''

\noindent
\textbf{Concepts introduced:} Activity Steps, Edges, Initial Nodes, Operations, and Activity Final

\noindent
\textbf{Operations provided:} sayHello

\subsubsection{Level 2}
Figure~\ref{fig:level2map} displays the map for level 2. The purpose of this level is to introduce the player to moving the game character around, and learn about timing delay with \emph{timers}. The goal is to make the character start moving forward, and then add correct timing so that the character will stop moving on top of the star, allowing him to pick it up. The time it takes to move from the starting position to the star is directly proportional to the distance, as it takes 500 milliseconds to cross one tile.

\begin{figure}[htp]
	\centering
	\includegraphics[scale=0.60]{level2map}
	\caption[Level 2 of the Reactive Blocks game]{Level 2 of the Reactive Blocks game. The player must make the character move 6 tiles forward by timing the movement correctly, and then pick up the star.}
	\label{fig:level2map}
\end{figure}

\noindent
\textbf{Goal:} Make the game character pick up the star

\noindent
\textbf{Concepts introduced:} Moving forward, Timers

\noindent
\textbf{Operations provided:} moveForward, stop, pickUp

\subsubsection{Level 3}
Figure~\ref{fig:level3map} displays the map for level 3. Modeling wise, this level does not introduce any new concepts, but lets the player become more familiar with timers, and use operations to also change the direction of the game character. Level 3 is a follow-up challenge for level 2.

\begin{figure}[htp]
	\centering
	\includegraphics[scale=0.50]{level3map}
	\caption[Level 3 of the Reactive Blocks game]{Level 3 of the Reactive Blocks game. The player must move the character around the maze, and pick up all the stars.}
	\label{fig:level3map}
\end{figure}

\noindent
\textbf{Goal:} Make the game character pick up all four stars

\noindent
\textbf{Concepts introduced:} Moving in different directions

\noindent
\textbf{Operations provided:} moveForward, stop, pickUp, turnLeft, turnRight, turnAround

\subsubsection{Level 4}
Figure~\ref{fig:level4map} displays the map for level 4. The purpose of this level is to introduce the player to \emph{decisions}, \emph{alternate branches}, and handling things that may not be known in advance. Players also learn to pass data from operations to decisions, allowing \emph{guards} to be set on outgoing edges.

\begin{figure}[htp]
	\centering
	\includegraphics[scale=0.45]{level4map}
	\caption[Level 4 of the Reactive Blocks game]{Level 4 of the Reactive Blocks game. The player must open the chest to reveal an either blue or yellow key, which allows one lock to be unlocked. The character can then pass through to the star.}
	\label{fig:level4map}
\end{figure}

\noindent
\textbf{Goal:} Open the chest to find a key, unlock the lock matching the key, and pick up the star

\noindent
\textbf{Concepts introduced:} Alternate branches, Decisions, Object flows

\noindent
\textbf{Operations provided:} moveForward, stop, pickUp, turnLeft, turnRight, turnAround, interact

\subsubsection{Level 5}
With the introduction of alternate branches, activity diagrams can become large and messy. Logic has to be (re)created for each branch, and if more branches follow, we quickly see an explosion of the decision tree. Fortunately, there may be ways of simplifying this, depending on what happens in each branch.

\noindent
Figure~\ref{fig:level5map} displays the map for level 5. There are 4 locks with different colors, meaning that there are 4 possible branches/paths after the chest has been opened. However, the final part of each path is identical, meaning we can use the same sequence to complete the logic. The purpose of this level is to introduce \emph{reuse} to players, by using the same sequence to complete all alternate branches with one or more \emph{merge} nodes.

\begin{figure}[htp]
	\centering
	\includegraphics[scale=0.50]{level5map}
	\caption[Level 5 of the Reactive Blocks game]{Level 5 of the Reactive Blocks game. The player must open the chest to reveal an either green, blue, orange, or yellow key, which allows one lock to be unlocked. The character can then pass through to the star.}
	\label{fig:level5map}
\end{figure}

\noindent
\textbf{Goal:} Open the chest to find a key, unlock the lock matching the key, and pick up the star

\noindent
\textbf{Concepts introduced:} Reuse of sequences, Merge nodes

\noindent
\textbf{Operations provided:} moveForward, stop, pickUp, turnLeft, turnRight, turnAround, interact

\section{Implementation}
\label{sec:game_implementation}
With the game concept in place, it was time to start working on the prototype implementation. Since Reactive Blocks is Java-based, the natural choice was to start with a Java framework for creating games. I settled for \emph{libGDX},\footnote{\url{http://libgdx.badlogicgames.com/}} a Java game development framework licensed under Apache 2.0.

\noindent
Even with a framework as a starting point, implementing the game prototype required considerable work. The game itself was implemented as a JAR to be included with a Reactive Blocks project containing blocks for each level. Graphics are based on or retrieved from various collections of free graphics on the web. For the interested reader, both the full source code\footnote{\url{https://github.com/Desarc/reactive-blocks-tutorials/tree/master/no.ntnu.oyvinric.tutorialgame}} and the Reactive Blocks project\footnote{\url{https://github.com/Desarc/tutorial-game}} is available on GitHub.

\noindent
The result is a game prototype with 5 levels, and an introduction part for each level. Below, each component is described in more detail, using level 4 as an example.

\paragraph{The Game Interface}
Figure~\ref{fig:level4} shows the interface of the game on a wide screen (1920x1080 pixels), with the Eclipse window running a Reactive Blocks perspective on the left, and the introduction window on the right. The player starts playing the game by ``building and running'' the \emph{Level1\_Introduction} block, which makes the introduction window pop up. The player can then open the \emph{Level1} block and start modeling, by right-clicking the empty canvas and adding elements. The required operations are already implemented, referencing the JAR with the game logic, and the player can simply add them to the model. An example solution for level 4 is displayer in Fig.~\ref{fig:level4_solution}. When the model is complete, it is time to ``build and run'' the \emph{Level1} block to see if the solution is correct. If this is the case, the player moves on to \emph{Level2\_Introduction} and repeats the process.

\begin{figure}[htp]
	\centering
	\includegraphics[scale=0.25]{level4}
	\caption[The Reactive Blocks game interface]{The Reactive Blocks game interface. On the right side is a regular Eclipse window running the Reactive Blocks perspective with the block for level 4 open. On the right side is the introduction window for level 4.}
	\label{fig:level4}
\end{figure}

\begin{figure}[htp]
	\centering
	\includegraphics[scale=0.50]{level4_solution}
	\caption[Level 4 solution]{An example solution for level 4 of the Reactive Blocks game.}
	\label{fig:level4_solution}
\end{figure}

\noindent
The introduction window is described in more detail in Sect.~\ref{sec:game_tutorial}, and matches the design mock-up. Like the rest of the game, it is implemented within the \emph{libGDX} framework.

\paragraph{The Game Window}
Figure~\ref{fig:game_window} shows the main game window after level 4 has been completed. This window is opened when the player ``builds and runs'' the \emph{Level4} block after completing the model, and assuming the model is correct, displays the desired behavior of the game character.

\begin{figure}[htp]
	\centering
	\includegraphics[scale=0.45]{game_window}
	\caption[Reactive Blocks game window]{The game window for the Reactive Blocks game after level 4 has been completed. The level map is displayer on the left, while the right side contains a \gls{hud} informing the user about the current status of the level.}
	\label{fig:game_window}
\end{figure}

\noindent
The main part of the window is the level map, where we can see the game character moving around and interacting with various objects. On the right side is a small \gls{hud}, which displays the current status of the level, such as how many stars have been picked up, how many need to be picked up in total, and which keys have been found.

\section{Usability Testing of the Game}
\label{sec:game_testing}
With a prototype of the game ready, it was time to do some usability testing to see how players are able to work with the \gls{ui}, and grasp the concepts presented sufficiently to complete the levels without help. Suitable test subjects were even harder to come by than with the tutorial in Ch.~\ref{ch:reactive_blocks_tutorial}, but I was able to recruit three volunteers. Two of the test subjects match part of the target audience, having some previous experience with programming and system engineering, while the third subject offers a slightly different perspective, being an experienced gamer.

\noindent
The primary goal for the usability test is to uncover issues about the game that may decrease the quality of the learning experience. Such issues may include poor affordance in the game \gls{ui}, or insufficient, missing, or confusing information about concepts.

\subsection{Testing Method and Collection of Results}
\label{sec:game_testing_method}
The testing session is carried out as a fairly standard discount usability test~\cite{nielsen:discount_usability}, with the significant difference that a functional prototype is used instead of a simple mock-up. The user is presented with some scenarios to go through, in this case starting with instructions on how to set up the game, followed by a challenge to solve for each level. While working on the tasks, the user is encouraged to think aloud in order to give the supervisor additional information about their choices.

\noindent
Supervisor intervention is kept to a minimum. The supervisor will only intervene when a problem with the game has been clearly established, and there is no point in watching the user struggling further.

\noindent
In an attempt to measure the user's understanding of the concepts presented in each level, the users are asked to fill in a feedback form (Appx.~\ref{appx:game_feedback_form}) rating their own understanding of these concepts after the level has been completed. The purpose of the questions in this form is to get an impression on how difficult the game is to understand, if there is anything missing from the introductions, and if the users feel that they have understood the concepts.

\subsection{Test Subject 1}
\label{sec:game_testing_subject1}
Test subject 1 is a 25 year old male with a university degree in petroleum engineering. The subject has some previous programming experience with numerical computations in MATLAB or Fortran, but no experience with system engineering. He is however a very experienced player of various video games within different genres.

\noindent
While observing test subject 1 playing the game, I discovered both some problems with the game \gls{ui} and some bugs that needed fixing. I will not go into details about the bugs, but attempt to highlight some of the \gls{ui} issues. Subject 1 spent a total of 83 minutes on the 5 levels of the game.

\noindent
The first thing I noticed was that the subject had trouble following the slide show, which would start automatically and change slide every 10 seconds. It is possible that having a slide show that runs automatically is not a very good idea, considering that users read and grasp concepts at different speeds.

\noindent
Throughout the whole game experience, the subject consistently had trouble navigating the Reactive Blocks \gls{ui}, making some tasks unnecessary difficult to complete, and slowing down the overall progress. This includes things like trouble finding the correct elements or trying to connect edges to nothing. At one point the subject accidentally double-clicked an operation, bringing up the Java code for the current level, which added some confusion.

\noindent
At level 1, the subject thought that an Activity Final node was required to complete the program. This was a completely logical observation, however unfortunate, since adding an Activity Final node to the application will terminate the game before the user can see that the level is completed. The tip that said an Activity Final node was not needed, was apparently not sufficient, as the subject eventually gave up on understanding what the problem was, and asked for help.

\noindent
In more than one case, the subject was able to complete a level without having correct timing. The game character would just keep walking into a wall until finally the timer expired, and a star was picked up.

\noindent
On level 4, the subject struggled with understanding decisions and guards, particularly which data type should be used. String values \emph{blue} and \emph{yellow} were mentioned in the introduction, but the subject believed the guards should be set to \emph{true} for blue, and \emph{false} for yellow. The subject was also confused by the option of adding an \emph{else} branch from decisions, which was not really necessary in this case. Additionally, it was not clear to the subject where the data would come from, and he started experimenting with \emph{variables}. In the end, the subject needed some help with understanding which data type should be set on the guards.

\noindent
While struggling with level 4, the subject started actively looking for ways to discover what the problem was, and discovered the \emph{analyze} functionality of Reactive Blocks. This could have helped him solve it, but he did not understand the error messages presented.

\noindent
After the subject had understood how to set guards from decisions in level 4, he noticed that the branches could be merged for the final part, which is the learning goal for level 5. Since merge nodes had not yet been introduced, the subject thought the \emph{join} node would be the correct choice, which resulted in some error messages the subject did not understand (in fact, these error messages did not really give any real information).

\noindent
When working on level 5, the subject tried to add multiple outgoing edges from an operation, and set guards on these, without adding a decision node in between. The subject also got confused by the \emph{connector merge} node in Reactive Blocks, as he tried to use this node instead of the regular \emph{merge} node, which he did not discover until later. Finally, the key/lock value corresponding to \emph{red} actually look orange within the game, which added some confusion.

\noindent
Each time the subject would get stuck or have trouble understanding a concept, he would check the tips for that level, or the help-on-demand resources for that concept. The subject needed extra information on levels 1, 4 and 5, but was able to complete 2 and 3 with the information from the introduction alone. Levels 2 and 3 were additionally completed on the first try, while the other levels required some trial and error.

\subsection{Test Subject 2}
\label{sec:game_testing_subject2}
Test subject 2 is a 24 year old male, working on a university degree in Computer Networking and Signal Processing. The subject has some previous programming experience from various university courses, ranging from basic object orientation to microcontroller programming, and some \gls{hci} implementation.

\noindent
Test subject 2's game sessions was also far from flawless. Subject 2 encountered many of the same problems with the game as subject 1, in addition to some new issues. Subject 2 actually uncovered more bugs during his session, because his approach to solving the levels touched more edge cases. Subject 2 spent a total of 66 minutes on the 5 levels of the game.

\noindent
First of all, subject 2 struggled with many of the same things related to the Reactive Blocks \gls{ui}. The difference was that subject 2 was familiar with some other software development tools, and thus had some expectations that were not met with the \gls{ui}. However, because of the initial awkwardness and annoyance experienced when working with the Reactive Blocks \gls{ui}, the subject later discovered some actions that would simplify the modeling process on his own, such as duplicating elements or sequences of elements.

\noindent
Like test subject 1, subject 2 also experienced confusion about the \emph{activity final} node. He commented that since the element had been introduced as a way of ``finishing'' a program, it was implied that this element should be the final part of all programs, and was actually \emph{required} to end an activity step.

\noindent
Test subject 2 additionally experienced some confusion about the goals for some levels, such as believing he had to return to the starting point in order to complete.

\noindent
Unlike subject 1, subject 2 was able to better understand decisions and guards, and was thus able to complete levels 4 and 5 more easily. However, since the \emph{object flow} was introduced as an individual concept, he believed this was an element separate from control flow edges, and spent quite some time looking for it.

\noindent
Likely due to his previous programming experience, the subject attempted to use the logical operation \emph{OR} to simplify the alternate branches in level 5. This is unfortunately not possible, as guards in Reactive Blocks only accept \emph{literal} values.

\noindent
Unlike subject 1, subject 2 was able to create the correct logic on his first try for all levels (disregarding the use of \emph{activity final} in level 1). Subject 2 also actively used the \emph{tips} functionality to find help, but rarely needed to check the help-on-demand resources.

\noindent
Subject 2 also mistook the red key/lock pair to be orange.

\subsection{Test Subject 3}
\label{sec:game_testing_subject3}
Test subject 3 is a 24 year old male, with roughly the same background as test subject 2.

\noindent
Like subject 2, subject 3 also uncovered quite a few bugs during his session, also some new ones. Otherwise, the issues encountered were largely the same as with subjects 1 and 2. Subject 3 spent a total of 74 minutes on the 5 levels of the game.

\noindent
With many of the same expectations for the Reactive Blocks \gls{ui} as subject 2, subject 3 also faced the same kind of awkwardness and annoyance when learning to work with it. Having already established that this was an issue, I gave the subject some tips on working with the \gls{ui} in order to lessen his frustration.

\noindent
Like the other two subjects, subject 3 found it natural to include an \emph{activity final} node to complete the program. He also explained that he did not realize the various parts of an activity step were performed without intermediate delay, and thus expected to see the result of completing the level even if the program was terminated (after a delay).

\noindent
Like subject 1, subject 3 was confused about how to add guards to decisions, not realizing he was supposed to first create outgoing edges. He attempted to work around this by using the \emph{Add else branch} option in Reactive Blocks, but this proved a more confusing approach, as it created \emph{control flow} edges which may only have \emph{else} guards.

\noindent
Unlike the other subjects, subject 3 preferred to solve some of the levels incrementally, particularly level 3 and 5, by completing part of the logic and checking if everything was correct so far. Subject 3 also spent more time reviewing \emph{tips} and help-on-demand sources than the other subjects. 

\noindent
Subject 3 also mistook the red key/lock pair to be orange.

\subsection{Data from the Feedback Forms}
\label{sec:game_feedback_data}
In addition to my observations, each test subject filled in the feedback form included in Appx.~\ref{appx:game_feedback_form}. This section summarizes the data gathered from these forms.

\noindent
The first question asked, for each level, if the introduction part gave enough information to solve that level. All three subjects answered consistently \emph{yes} for all levels, except subject 2 for level 1. Subject 2 further commented that he thought the introduction part gave misleading information about the \emph{activity final} node, causing his solution to level 1 to sort of be both right and wrong at the same time. It was difficult to understand what was going wrong, and subject 2 eventually had to ask for help from the supervisor. As we can see from my observations in the previous sections, this actually also happened with the two other subjects. Since subjects 1 and 3 answered that the introduction part gave them enough despite their trouble, they may have considered this as an oversight of their own, rather than missing information.

\noindent
Question 3 asked the subjects to rate how difficult they found each level to be, and the results are displayed in Tab.~\ref{tab:subject_difficulty}. From the table, we see that all three subjects found levels 2 and 3 to be easier than the others. The two subjects with more diverse programming backgrounds, subject 2 and 3, also consistently experienced each level as being easier than subject 1.

\begin{table}[htp]
	\centering
	\begin{tabulary}{\textwidth}{| C | C | C | C |}
		\hline
		\textbf{Level} & \textbf{Subject 1} & \textbf{Subject 2} & \textbf{Subject 3} \\
		\hline
		1 & Difficult & OK & OK \\
		\hline
		2 & OK & Easy & Easy \\
		\hline
		3 & OK & Easy & Easy \\
		\hline
		4 & Difficult & OK & OK \\
		\hline
		5 & Difficult & Easy & Easy \\
		\hline
	\end{tabulary}
	\caption[Test Subject 1 Perceived Difficulty]{The \emph{perceived level of difficulty} for each level and each test subject. Available options were in ascending order \emph{Too Easy}, \emph{Easy}, \emph{OK}, \emph{Difficult}, and \emph{Too Difficult}.}
	\label{tab:subject_difficulty}
\end{table}

\noindent
Question 4 asked the subjects to rate their own level of understanding for each concept presented, and the results are displayed in Tab.~\ref{tab:subject_understanding}. There is a lot of variation, but we should notice that none of the subjects felt they gained a \emph{complete} understanding of \emph{activity steps}, \emph{activity final}, and \emph{timers}. Subjects 1 and 3 actually did not rate their own understanding as complete for any of the concepts presented.

\begin{table}[htp]
	\centering
	\begin{tabulary}{\textwidth}{| C | C | C | C |}
		\hline
		\textbf{Concept} & \textbf{Subject 1} & \textbf{Subject 2} & \textbf{Subject 3} \\
		\hline
		Activity Steps & Partly & Mostly & Mostly\\
		\hline
		Edges & Mostly & Completely & Partly \\
		\hline
		Operations & Mostly & Completely & Mostly \\
		\hline
		Initial Nodes & Mostly & Completely & Mostly \\
		\hline
		Activity Final & Partly & Not at all & Mostly \\
		\hline
		Timers & Mostly & Mostly & Mostly \\
		\hline
		Decisions & Partly & Completely & Mostly \\
		\hline
		Object Flow & Mostly & Completely & Mostly \\
		\hline
		Merge & Mostly & Completely & Mostly \\
		\hline
	\end{tabulary}
	\caption[Test Subject 1 Levels of Understanding]{The self-rated \emph{level of understanding} for various Reactive Blocks elements by each test subject. Available options were in ascending order \emph{Not at all}, \emph{Partly}, \emph{Mostly}, and \emph{Completely}.}
	\label{tab:subject_understanding}
\end{table}

\noindent
The fifth question asked, for each level, whether the subjects thought they would be able to use the concepts they had just learned about to solve other problems. All three subjects consistently answered \emph{yes} to this question for all levels.

\noindent
In addition to the question asked for each level, there were some final questions. The first question asked whether the subjects \emph{enjoyed} playing the game, to which subjects 1 and 3 answered \emph{yes}, and subject 2 answered \emph{a little}. As a follow up, they were asked to comment on what they did or did not like. While subject 1 left this field empty, subject 2 mentioned that it was a good way of learning concepts, and that it gave him a sense of achievement. Subject 3 mentioned that the \gls{ui} had been somewhat annoying to work it. Finally, the subjects were asked if they would prefer the game as a way to learn about \gls{uml} Activities and Reactive Blocks, compared to other options. All three subjects answered in favor of the game.

\subsection{A Side-note on an Informal Experiment}
\label{sec:game_testing_sidenote}
In addition to the formal usability tests conducted with the three volunteer subjects, I thought it might be interesting to test the game experience on someone completely outside the target audience, just to see what happened. I decided my girlfriend, an elementary school English teacher with no experience even remotely related to programming (she does not like working with computers), would be a suitable subject.

\noindent
The experiment was conducted in an informal way, where she would simply do her best to absorb the information provided in the introductions, and then try to solve the exercises. I helped with some of the parts that were more technical and less relevant, such as preparing Eclipse and Reactive Blocks, and building and running the blocks. If she got completely stuck, I would also offer some hints, primarily about where she could find more information.

\noindent
To my slight surprise, my girlfriend actually handled the problems presented very well. She read the introductions carefully, and easily understood the concepts of operations, tokens and control flow. She handled timers without trouble, struggled a little with decisions (I had to explain what \emph{Strings}, \emph{booleans}, and \emph{integers} were), and used \emph{Merge} nodes perfectly to simplify the logic in level 5. She did spend notably more time on each level than the three subjects in the formal tests (she even got a little more help with \gls{ui} issues), but this could simply be because she was not used to building this particular kind of mental models, added to the fact that she read the instructions extra carefully.

\noindent
Despite her aversion for working with computers and lack of interest for software development, she found the game fun to play because it let her get the sense of mastering a skill she thought was far beyond her abilities. When starting both level 4 and 5, she sighed and exclaimed that \emph{``it looks really difficult''}, but attacked the problems with determination, reading and re-reading the introductions until things started to make sense.

\noindent
I should also include that while my girlfriend was able to understand the concepts presented sufficiently to solve problems within the context of the game, she admitted to having no idea about how they could be used for other purposes.

\noindent
While I found this experiment to be quite interesting, its informal nature prevents me from drawing any real conclusions. It is likely that a large part of my girlfriend's motivation was to support my thesis work, as opposed to wanting to learn about \gls{uml} Activities. Also, despite not having any interest in working with computer-related topics, she could simply be the type of person who has a knack for understanding these things (without even knowing it). All in all, much is left to speculation, but the experiment provides an interesting anecdote for the potential of educational games.

\section{Evaluation of the Tutorial Game}
\label{sec:game_evaluation}


