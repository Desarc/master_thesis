\noindent
When learning to use a software modeling language, people are usually directed towards the language's documentation, or ``language tutorials'' in the form of a set of instructions required to get started with the language. Most of the information about the language is generally delivered early in the process, requiring the developer to either understand (and remember) a lot of concepts at once, or to go back and search for the information when it's needed.

\noindent
On the contrary, many computer games come with no instructions at all, but rather start by guiding the player through a set of training levels. Each level introduces a new element, concept or strategy, and sometimes let the player experiment with the new concepts to solve more advanced problems.

\noindent
In this thesis, it will be examined to which degree this approach can be used for introducing and learning a new modeling language. Methods for creating more immersive tutorial experiences with required information delivered in-context will be explored and tested, using UML Activities and the Reactive Blocks tool as an example. The thesis will also explore whether the context of the tutorial matters, and if there are possible advantages gained by designing the tutorial experience itself as a game.