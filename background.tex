\chapter{Background}
\label{ch:background}
This chapter contains the background material that serves as the basis for this thesis. The following sections are a mix of summarizations of other people's work, and some gathering and analysis of information done by me.


\section{Software Modeling and Modeling Languages}
\label{sec:software_modeling}
In the world of software development, potential problems and challenges that may arise when developing a product are plentiful.


\subsection{The Purpose of Software Modeling}


\subsection{Modeling Languages}


\subsection{Automatic Code Generation}


\section{Tutorials}
\label{sec:tutorials}
``A tutorial is a method of transferring knowledge and may be used as a part of a learning process. More interactive and specific than a book or a lecture, a tutorial seeks to teach by example and supply the information to complete a certain task.''~\cite{wiki:tutorial}

\noindent
Tutorials are often used to teach and introduce new concepts and topics to users who previously have little or no experience with it. Tutorials may be designed for learning a vast range of topics and concepts, such as:
\begin{itemize}
	\item Programming, or using specific programming or modeling languages.\footnote{\url{http://docs.oracle.com/javase/tutorial/}, \url{http://www.tutorialspoint.com/uml/}}
	\item Spoken languages.\footnote{\url{http://ielanguages.com/}}
	\item Software products.\footnote{\url{http://www.photoshoptutorials.ws/category/photoshop-tutorials/}}
	\item Video games (such tutorials are usually presented inside the game itself)
	\item Real-life skills, like photography.\footnote{\url{http://photography.tutsplus.com/}}
	\item Human sciences.\footnote{\url{http://anthro.palomar.edu/tutorials/}}
\end{itemize}

\noindent
The above list is in no way exhaustive, but meant to offer some examples of the numerous and diverse topics that may be introduced with the help of a tutorial.

\noindent
Additionally, tutorials come in many forms. The most common form of tutorial is likely the text-based tutorial, often supplemented by illustrations and pictures, but tutorials also come in the form of videos, animations, audio, or in the case of many video games, an interactive experience combining any or all of these.

\subsection{The Structure of a Tutorial}
\label{sec:tutorial_structure}
Tutorials for different types of topics are generally structured in a way the author believes will provide a good introduction for the given topic, starting with the necessary basic information, and then building on this to learn more advanced concepts. Depending on both the topic and the author, this may result in very different structures.

\noindent
Looking at some of the examples from Sect.~\ref{sec:tutorials}, we see that while the Java tutorials provide stepwise instructions for reaching a specific goal, the spoken language tutorials serve as more of a lookup reference for the most basic concepts within the language. The spoken language tutorials are actually in some ways similar to the separate Java API documentation.\footnote{\url{http://docs.oracle.com/javase/8/docs/api/index.html}}

\noindent
While there are differences in how tutorials for various topics are made, we may infer some general patterns and elements present in a wide range of different tutorials.

\begin{itemize}
	\item Basic and/or advanced information about the topic, depending on the scope of the tutorial. Often presented in a stepwise manner, starting from the most basic and moving on to the more advanced.
	\item Examples on how to use the information provided in specific cases or contexts.
	\item Exercises where the reader must try to use the concepts introduced in a specific context.
	\item Illustrations, figures, or animations, providing the reader with additional examples, information about concepts, or desired results from exercises.
	\item Some kind of motivation for learning about the topic, often as part the tutorial introduction.
\end{itemize}

\noindent
How these patterns and elements are presented depends on the form of the tutorial.

\subsection{The Characteristics of a Tutorial}
\label{sec:tutorial_characteristics}


\subsection{What Makes a Tutorial Good?}


\section{Tutorials for Programming and Software Modeling}
\label{sec:programming_tutorials}
In order to teach potential users to use a specific programming language, paradigm, protocol or modeling language (from now on referred to simply as a \emph{topic}), various additional tools or sources of information may be provided. Generally, some form of formal documentation or specification of the language or protocol is considered mandatory, and serves as the \emph{definitive} source of information. Additionally, various tutorials and exercises are often created in order to provide a better introduction to the topic, decreasing the threshold for learning this topic.

\subsection{Documentation}
While a topic's documentation often serves as its primary source of information, such formal documents are not necessarily the best source of information when \emph{learning} about the given topic. Often it is not intended as a learning resource, but rather as a formal reference for users already familiar with the topic. In the documentation, the topic is generally presented in a structured way with respect to its various aspects, so that it is easy for someone already familiar with the topic to find the required information.

\noindent
A good example of this is the \gls{uml} specification.\footnote{\url{http://www.omg.org/spec/UML/2.4.1/}} While the specification offers detailed information about every aspect of \gls{uml} in a way that suits an experienced user, it is likely confusing and not particularly helpful for a user with no previous experience or knowledge about \gls{uml}. Using this specification as a starting point for learning \gls{uml} is likely to require a lot of effort from the user. In the worst case, the user may not even learn all the concepts properly, despite the provided information being very specific and accurate.

\noindent
More importantly, the user may not learn how to properly apply the learned concepts to a given situation. In many cases, 


\subsection{Advantages and Challenges}




\section{Game-based learning}
\label{sec:game_based_learning}

\subsection{Gamification}


\subsection{Learning Inside the Game}


\subsection{Applying Acquired Knowledge and Skills in the Game}




\section{Tutorials in Games}
\label{sec:tutorials_in_games}


\subsection{Purpose}

\subsection{Tutorial Characteristics and Their Importance}

presence, context-sensitivity, freedom, availability of help-on-demand




